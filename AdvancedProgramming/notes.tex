\documentclass[12pt]{article}
\usepackage{listings}
\usepackage{color}
\usepackage{amsmath}
\usepackage{lmodern}        % Latin Modern family of fonts
\usepackage[T1]{fontenc} 

\definecolor{dkgreen}{rgb}{0,0.6,0}
\definecolor{gray}{rgb}{0.5,0.5,0.5}
\definecolor{mauve}{rgb}{0.58,0,0.82}
\definecolor{backcolor}{rgb}{0.95,0.95,0.95}

\lstset{
	frame=tb,
	language=bash,
	aboveskip=3mm,
	belowskip=3mm,
	showstringspaces=false,
	columns=flexible,
	basicstyle={\small\ttfamily},
	% numbers=left,
	% backgroundcolor=\color{backcolor},
	numberstyle=\tiny\color{gray},
	keywordstyle=\color{blue},
	commentstyle=\color{dkgreen},
	stringstyle=\color{mauve},
	breaklines=true,
	breakatwhitespace=true,
	tabsize=3
}

\title{Advanced C Programming}
\author{Rodrigo Rodrigues}
\date{12-08-24}

\begin{document}
\maketitle
\newpage
\section{Examples}
\begin{lstlisting}[language=C, title=size of]
	int main(){
		if (sizeof(int) > -1)
			printf("True\n");
		else
			printf("False\n");
	}
\end{lstlisting}
Output: \textbf{False}

Int has a size of 4, but comparing only works on the same data type.
Sizeof(int) will be unsigned int, while -1 is a signed int. So by treating -1 as an unsigend value -1=0xFFFFFFFF.

\vspace{10mm}

\begin{lstlisting}[language=C, title=float vs double]
	int main(){
		float f = 0.1;
		if( f == 0.1)
			printf("True\n");
		else
			printf("False\n");
	}
\end{lstlisting}
Output: \textbf{False} 

Double data-type has more precision than float because C is oriented to work with doubles. Float as 6 degress and double has 10 degrees. If we treat f as a double, f will be converted into a double, and its not going to havethe same value.

\vspace{10mm}
\newpage

\begin{lstlisting}[language=C, title=Sizeof2]
	int main(){
		int a, b = 1, c = 1;

		a = sizeof(c = ++b + 1);
		printf("a = %d, b = %d, c = %d\n, a, b, c");
	}
\end{lstlisting}
Output: \textbf{a=4, b=1, c=1} 

Sizeof is a runtime compilation, \textbf{a} is going to change, but not \textbf{b} or \textbf{c}.

\vspace{10mm}

\begin{lstlisting}[language=C, title=charptr]
	int main(){
		char *p = 0;

		*p = 'A';
		printf("Value at p = %c\n", *p);
	}
\end{lstlisting}
Output: \textbf{Segmentation fault (core dumped)}

We initialize a pointer in the memory address 0, which is read only. When we try to write A to it we get a segmentation fault.

\vspace{10cm}

\begin{lstlisting}[language=C, title=nested if]
	int main(){
		int a = 1, b = 3, c = 2;
		if (a>b)
			if(b>c)
				printf("a>b and b > c\n");
		else
			printf("something else\n");
	}
\end{lstlisting}
Output: \textbf{}

Nothing will be printed, the else statement bellongs to the second if.

\newpage

\section{Compiler}
C is a robust language (kernel is writen in C i think) and we can write everything from simple to complex program.

\vspace{2mm}

C compiller allows o write both assembly language programming allong side C, allowing you to write both system software and application software. 

\vspace{2mm}

Programming in C is efficient and fast. Compiling it produces object code which is linked by linker, and a binary executable code is generated. 

\vspace{2mm}

The \textbf{main()} function is the entry point.

\vspace{2mm}

Any \textbf{variable} needs to be written in lowercase letters.

\newpage

\section{Data Types}

\subsection{Primitive type}%
\label{sub:Primitive type}
Char is a byte (8bits).
if we give char = 'A', we are not storing A, we store 65, 0b010001

\begin{enumerate}
	\item char - 1 bytes (8 bits).

	\item short (short integer) - 2 bytes (16 bits).

	\item int - 4 bytes (32 bits).
		\begin{enumerate}
			\item for 32 enviroment
		\end{enumerate}

	\item long (store integer) - 4 bytes (32 bits). Depends on the platform.
		\begin{enumerate}
			\item its for 64 enviroment
			\item short, int and long all store integers, but for diferent objectives
		\end{enumerate}
	\item float - 4 bytes (32 bits) 6 digits of precision

	\item long long - 8 bytes (64 bites)
		32 bit enviroment;\\
		C compiler unnderstands it.\\
		we can store 4GB

		sometimes we have 100Gb, 200GB or 300TB in our system, so:\\
		int = 4 byte\\
		long = 4 byte\\
		none of this wew enoughf\\
		long long was necessary\\
		long long = 8 bytes

		In a 64 enviroment, it's just a long.
	\item double - 8 bytes (64 bits).
		\begin{enumerate}
			\item for 32 processors
			\item removes the dependency from shot, int, long
			\item to store more precision (10 digits of precision) compared to float.
		\end{enumerate}

	\item void - 4 bytes (it means nothing, we do not alocate memory)
		Created for pointers.
		\begin{enumerate}
			\item created for functions that need to return nothing
			\item usefull for functions that return something we dont know
			\item for example malloc() returns a piece of memory (pointer [has no data type])
		\end{enumerate}
		Integral, real and character types can be signed or unsigned

		\subsection{Example}%
		\label{sub:Example}

		\textbf{Which data type is most suitable to store size of a hard disk partition (10 GB) ? }


	\begin{lstlisting}[language=C, title=example of memory alocation]
		// WARNING: wrong
		int sizeofdisk;
		// WARNING: wrong
		unsigned int;
		// Correct
		unsigned long;
		sizeofdisk = 10GB = 10 * 1024 * 1024 * 1024 * 1024
	\end{lstlisting}

	\vspace{10mm}

	GB = * 1024 * 1024 * 1024 * 1024 \\
	1 KB = 1024 B; = $2 ^ {10}$ \\
	1 MB = 1024 KB; \\ 
	1 GB = 1024 MB; 8 000 000 000 bites

	\vspace{10mm}


	integer: 4 bytes: 32-bits: max value = $2^32 - 1$ 

	(msb) 11111111 11111111 11111111 11111111 (lsb)

	\begin{align}
		&2 ^ {32} = 2 ^2 * 2 ^ {30} \\
		&= 4 * 2^{10} * 2^{10} * 2^{10}\\
		&= 4 * 1024 * 1024 * 1024\\
		&= 4 * KB * KB * KB \\
		&= 4 GB
	\end{align}

	\newpage

	\textbf{What data type to store data of 8-bit microprocessor?}

	Unsigned Char - 8 bits

	it doesnt make sense to send a negative number to a controler \\ for exmaple, 
	so we dont use char. \\
	0000 0000 - $ 2^ 8$ functions (32)

	\vspace{10mm}

	\textbf{Which data type to store the length of a string (C strlen() function) ?}

	unsigned long - i guess not (but close)

	\begin{enumerate}
		\item can it be negative ?\\
			no
		\item how big its it?\\
			R.: unsigned int / unsigned long
			i dont know
	\end{enumerate}

	strlen() returns 'size\_t', which is not a primitive type.
	we use 'typedef'

	\begin{lstlisting}[language=C, title=defining size of an object]
		typedef unsigned long size_t;
\end{lstlisting}


\end{enumerate}

\subsection{User defined type}%
\label{sub:User defined type}

\begin{enumerate}
	\item Defined by Programmer
	\item structure
	\item union
	\item enumeration
	\item typedef

\end{enumerate}

\vspace{10mm}

\begin{lstlisting}[language=C, title=defined types]

struct empdb {
  int empno;
  char name[30];
  int dept_id;
};

union u1 {
  int a;
  int b;
};

struct empdb emp;
\end{lstlisting}


dont define a structure as a union, and dont define a union as a structure, they are for diferent things

The defined type will hold the memory for the biggest type, which is int. R.: 4 bytes

Union is for something that has "versions", more than one choice, i guess

\newpage

\begin{lstlisting}[language=C, title=code]

#include <stdio.h>
struct empdb {
  int empno;
  char name[30];
  int dept_id;
};

union u1 {
  int a;
  int b;
};

int main(){

  long long l;
  long sl;

  l = 0xAABB CCDD EEFF 00;
  // in a 32 bit enviroment long long is meaninfull
  sl = 0xAABBCCDDEEFF00;
  // long (sl) is only 4 bytes in a 32 env
  printf("value of l = 0x%llx\n", l);
  printf("value of sl = 0x%lx\n", sl);
}
\end{lstlisting}

\newpage

\subsection{Derived/Dependent type}%
\label{sub:Derived/Dependent type}

\begin{enumerate}
	\item arrays
	\item pointer
	\item function pointer
	\item dependent on other types
	\item array of char
	\item pointer to char
	\item pointer to pointer
\end{enumerate}


\newpage

\section{Memory allocation}
\indent
\textbf{What is the memory size of char, int, long type?} \\
char - 1, int - 4, long - 8

\bigskip

\textbf{What is the size of memory allocated by the compiler for char, int and long pointer in a 32-platform?}

??? no idea


\vspace{10mm}
\textbf{What is the size of memory allocated for char, int, long pointer for 64 platform}

????


\vspace{10mm}

\textbf{What is the size of memory allocated by the compiler for void datatype and void pointer?}

????

\vspace{10mm}

\textbf{What is the output of the following in 64 plat?}

\begin{lstlisting}[language=C, title=sizeof cenas]
int main(){
	char c;
	printf("Sizeof c = %d, %d \n", sizeof(c), sizeof(char));
}
\end{lstlisting}
R.: 1, 1

\newpage

\textbf{What is the output of the following in 64 plat?}

\begin{lstlisting}[language=C, title=sizeof cenas]
int main(){
	char *cp;
	printf("sizeof cp= %d, %d\n", sizeof(*cp), sizeof(char*));
}
\end{lstlisting}
R.: 1, 8

\textbf{What is the output of the following in 64 plat?}

\begin{lstlisting}[language=C, title=sizeof cenas]
int main(){
	void v, *vp;
	printf("sizeof cp= %d, %d\n", sizeof(v), sizeof(vp));
}
\end{lstlisting}
R.:compilation error

\subsection{Why?}
We need to understand how many bytes a union or a strcture is because it will be helpfull when working with the user space debugging.  Whenever we get a core dumbped we need to understand what is causing it.

We need to unserdatna  a kernel space debuggin

Crash analysis or a memory dumps. We need to look at the values and understand what they mean.

\subsection{Memmory alocation - Primirive Data Types}

\begin{enumerate}
	\item char - 1 bytes (8 bits).

	\item short (short integer) - 2 bytes (16 bits).

	\item int - 4 bytes (32 bits).

	\item long (store integer) - 4/8 bytes. Depends on the platform.

	\item float - 4 bytes (32 bits) 6 digits of precision

	\item void -  it means nothing, we do not alocate memory.
\end{enumerate}

\subsection{Unsigned vd Signed}
The Bytes per data type do not change, but the value it can store differs. For example:\\ Char is 8 bytes storing values from -128 to 127. \\Unsigned char has 8 bytes as well, but stores values from 0 to 255. \\This happens to all the data types as far as i know.

\vspace{10mm}

Long long Type - created for 32 env, 4 bytes, 4 bytes [8 bytes 32 bit env].
use outside of 32 bit env. In a 64 env, it's viewed as a long.

Pointer is 4 bytes, long is 4 bytes long as well.

\newpage

\section{Data Types - Bits and Bytes}

\begin{lstlisting}[language=C, title=testing data types]
#include <stdio.h>
int main(void) {
  unsigned char c;
  printf("c = 255, displayed value c = %d\n", c = 255);
}
\end{lstlisting}
If we had just a char, output would be -1, because char only stores values ranging from -128 to 127 (0xFF). \\ But because its a unsigned char and doesnt need to store a signal, the we can store up to 255 and prints the correct value.


\begin{lstlisting}[language=C, title=code]

  int i;
  printf("i = -2, displayed value i = %x\n", i =- 2);

\end{lstlisting}
For some reason, its not printed 2, but fffffffe. If i had to guess i would say its because ffff fffe is how you display negative numbers in hex, but i dont know why its beeing displayed in hex. \\It turns out that '\%x' is how you display hexadecimal numbers.



\vspace{20mm}

\newpage

\title{\textbf{ScratchPad}}\\
8 bit value of 7\\
0000 0000 -\> 8 bits\\
0000 0111 - 7 in binary unsigned or signed\\
---------------------------------------------------------------------------------------------------\\ 
but what if we whant -7 in binary, now we have to use signed values\\
to do that we change the binary value we have to 2's complement:\\
---------------------------------------------------------------------------------------------------\\ 
0000 0111 - 7 in binary
\textbf{invert bits}\\
1111 1000\\
\textbf{add 1 in the end}\\
1111 1001 - -7 in binary signed

\end{document}

